\documentclass[12pt]{article}
\usepackage{amsmath}
\newcommand{\myvec}[1]{\ensuremath{\begin{pmatrix}#1\end{pmatrix}}}
\newcommand{\mydet}[1]{\ensuremath{\begin{vmatrix}#1\end{vmatrix}}}
\newcommand{\solution}{\noindent \textbf{Solution: }}
\providecommand{\brak}[1]{\ensuremath{\left(#1\right)}}
\providecommand{\norm}[1]{\left\lVert#1\right\rVert}
\let\vec\mathbf
\title{Linear Equations in Two Variables}
\author{Aryam Agrawal (aryamagrawal@sriprakashschools.com)}
\begin{document}
\maketitle
\section*{10$^{th}$ Maths - Chapter 3}
This is Problem-2.3 from Exercise 3.2
\begin{enumerate}
\item 
\end{enumerate}
\solution\\
Given Data:6x - 3y = -27\\ 
           2x - y = -9\\
This can also be written as:
\begin{align}
\myvec{6&-3&-27\\2&-1&-9}
\end{align}
now,Making $R_2 \xrightarrow\ R_1 - 3R_2$\\ 
we get,
\begin{align}
\myvec{6&-3&-27\\0&0&0}
\end{align}
Since, we are getting zero in $R_2$\\
It is a dependent equation.
\end{document}