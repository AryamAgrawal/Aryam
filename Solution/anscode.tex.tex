\documentclass[10pt]{article}
\usepackage{amsmath}

\newcommand{\myvec}[1]{\ensuremath{\begin{pmatrix}#1\end{pmatrix}}}
\newcommand{\mydet}[1]{\ensuremath{\begin{vmatrix}#1\end{vmatrix}}}
\newcommand{\solution}{\noindent \textbf{Solution: }}
\providecommand{\brak}[1]{\ensuremath{\left(#1\right)}}
\providecommand{\norm}[1]{\left\lVert#1\right\rVert}
\let\vec\mathbf
\title{Linear equations in 2 variables}
\author{Aryam (aryamworks17@gmail.com)}
\begin{document}
\maketitle
\section*{Class 10$^{th}$ Maths - Chapter 3}
This is Problem-2.3 from Exercise 3.2
\begin{enumerate}
\item On comparing the ratios $\frac{a_1}{a_2}$ , $\frac{b_1}{b_2}$ ,$\frac{c_1}{c_2}$, find out whether the lines representing the following pairs of linear equations intersect at a point, are parallel or coincident:\\
\begin{align}
9x+3y+12=0\\
18x+6y+24=0
\end{align}
\solution \\
Equations can be written as:\\
\begin{align}
\mydet{9&3\\18&6} \mydet{x\\y}= \mydet{-12\\-24}\\
 x= \frac{\brak{\vec{B}\times\vec{a_2}}}{\vec{a_1\times a_2}}
=\frac{\mydet{-12&3\\-24&6}}{\mydet{9&3\\18&6}}
=\frac{{(-12)(6) - (-24)(3)}}{{(9)(6) - (18)(3)}}
=\frac{-72+72}{54-54}
=0\\
 y= \frac{\brak{\vec{a_1}\times \vec{B}}}{\vec{a_1\times a_2}}
=\frac{\mydet{9&-12\\18&-24}}{\mydet{9&3\\18&6}}
=\frac{9(-24) - 18(-12)}{9(6) -18 (3)}
=\frac{-216+216}{54-54}
=0\\
\end{align}
Hence this equation has infinite number of solutions
\end{enumerate}
\end{document}

